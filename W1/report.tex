%
% report.tex
%
% Assignment 1, ITP
% by Casper B. Hansen, fvx507@alumni.ku.dk
%

\documentclass[11pt]{article}

\usepackage[utf8]{inputenc}
\usepackage{multicol}

\newcommand{\colbreak}{\vfill*\columnbreak}

\title
{
    Assignment 1\\
    {\large IT Project Management}
}

\author
{
    Casper B. Hansen\\
    Department of Computer Science\\
    University of Copenhagen\\
    {\tt fvx507@alumni.ku.dk}
}

\begin{document}

\maketitle
\begin{multicols}{2}
    \begin{abstract}
    In this assignment we go over identifying criteria for a succesful project,
    taking on the hypothetical role as the project manager for `Det Digitale
    Tinglysningsprojekt`-project.

    We will identify some succes criteria, argue their value to the company
    and device a process from which we can reach realistic goals and come to
    an agreement on.

    Lastly, we will describe what `Measurable Organisational Value` means, and
    how it is developed in practice.
    \end{abstract}
\colbreak
    \tableofcontents
\end{multicols}
\thispagestyle{empty}
\clearpage

\section{[A] Identifying \& Arguing Criteria for Success}
First, we will identify five criteria for a succesful project, if I was the
project manager, for this particular project.

\subsection{Chosen Success Criteria}
\begin{description}
    \item[1] If the project is succesful we will see a decrease in manual
        labor by at least \%50 within the first year of its operation.
    \item[2] If the project is succesful we will see a decrease in cost of
        operation by at least 88,3 mio. DKR. yearly, measurable 3 months
        after the system is brought online.
    \item[3] If the project is succesful, no application which this project
        is intended to handle, will involve any paper.
    \item[4] If the project is succesful the system should not be required
        to change existing systems to facilitate it --- it should integrate
        naturally into existing systems.
    \item[5] If the project is succesful applicants should be able to use the
        solution by way of `Digital Signatur`, and nothing else.
\end{description}

\subsection{Argumentation}
I argue that (1) should be set lower than the projected expectation from the
prior analysis, as new systems, and especially those in the public sector, have
proven to be difficult to implement and set in motion. Also, unexpected
problems can arise during such a shift of operations. Therefore the goals are
set to reflect a realistic point of view, rather than an optimistic one.

The projection of (2) is rather straight forward, as it is primarily job
lay-offs, which I consider to be quite easy to determine ahead of time,
therefore this is kept a strict requirement for the project.

In the documents (3) is rather clearly defined as an objective the project
seeks to solve. This should be a strict constraint.

The among the many parties affected by the introduction of this system, there
is a clearly communicated goal of integrating the system well with existing
systems --- although somewhat vague or implicit. Point (4) puts this forth as
a strict constaint.

The usability of the system is of utmost importance, and because its user base
is so large, I argue from (4) that we should enforce and constrain the project
to the use of `Digital Signatur` authentication system, as the target audience
is familiar with it. If we impose other or additional systems, the project will
suffer a decline in projected usage, which would affect all other variables.

\subsection{Ensuring acceptance and realistic criteria}
I would present the five criteria above to the stakeholders and argue, why they
have been either loosened or kept as a strict constraint. This gives the
stakeholders some measure of realistic goals, that provide a tangible foundation
which we can then pursue. Furthermore, I would use the MOV methodology, as
described in the following section, to make sure that every criteria reflects
the stakeholders interests and values, while maintaining a realistic and
clearly definable goal.

\section{[B] Measurable Organisational Value (MOV)}
In short, `Measurable Organisational Value`, or MOV, provides the guidelines
for which the solution seeks to achieve. It is second step, and a critical part
of building a so-called `business case`.

\subsection{Characteristics}
The MOV should encompass the following traits and characteristics.
\begin{description}
    \item[Measurable]
        means that we can compare the defined metric in the final product with
        the projected value expectation at the evaluation stage of the product.
\end{description}

\subsection{Development}
In practice, a MOV is developed by the following;
\begin{description}
    \item[Desired Area \& Impact] is defined in dialogue with the client. The
        client expresses how the initial idea came to be, and the reasoning
        behind it. From this we, as project managers, gain an understanding of
        the nature of project, as well as necessary background information,
        motivation and how the project is expected to impact the organisation.
    \item[Desired Value] is defined by asking interrogative questions, rooted
        in the the former step. These questions should be characterised by the
        use of comparative adjectives (e.g. better, faster, cheaper). This
        ensures that we have a clearly defined goal of value, as well as
        provides a solid foundation for the following step. In other words,
        how is the former step translated into measurable/comparable values.
    \item[Metric] is a collection of affected business variables, that the
        project will either contribute positively and negatively to. The
        former step supports this by the comparative questionaire form, as
        comparatives are by nature measurable by increase or decrease. This
        step allows stakeholders to determine the viability of, and risks
        associated with the project.
    \item[Time Frame] puts additional contraints on the project, in the form
        of returned value as a function of time (i.e. an increase in customers
        by at least \%5 six months from the system goes online). Make note that
        this is a not a projected expectation, but a constraint that the
        project must deliver.
    \item[Agreement] From the previous steps everyone involved with developing
        the MOV, including stakeholders and project manager, must review the
        MOV to ensure that it accurately describes the intent and that it is
        indeed a realistic pursuit. All parties must verify and agree to this.
    \item[Summarize] From the agreement, a concise statement or figure is
        produced that summarizes the goal of the project. In example, one or
        more sentences that describes the achievement sought to accomplish,
        like `The project is succesful if we see a \%5 increase in customers
        by the end of the first six months of its operation`.
\end{description}

\end{document}
